\documentclass[conference]{IEEEtran}
\usepackage{cite}
\usepackage{amsmath,amssymb,amsfonts}
\usepackage{algorithmic}
\usepackage{graphicx}
\usepackage{textcomp}
\usepackage{xcolor}
\usepackage{listings} % For code snippets
\usepackage{hyperref}

\begin{document}

\title{Agri-Intel: District-Wise Yield Optimization System using Heterogeneous Data Sources}

\author{\IEEEauthorblockN{Student Name 1}
\IEEEauthorblockA{\textit{Dept. of Computing} \\
\textit{NUST-SEECS}\\
ID: 123456}
\and
\IEEEauthorblockN{Student Name 2}
\IEEEauthorblockA{\textit{Dept. of Computing} \\
\textit{NUST-SEECS}\\
ID: 654321}
\and
\IEEEauthorblockN{Student Name 3}
\IEEEauthorblockA{\textit{Dept. of Computing} \\
\textit{NUST-SEECS}\\
ID: 789012}
}

\maketitle

\begin{abstract}
% Write 150 words summarizing:
% 1. The problem (Inefficient fertilizer use in Punjab).
% 2. The solution (ML model integrating Satellite, Climate, and Govt data).
% 3. Results (RMSE score, key findings).
% 4. The PoC (Interactive Streamlit Dashboard).
\end{abstract}

\section{Introduction}
% Discuss the importance of Food Security (Theme T4).
% Mention that fertilizer misuse leads to poor yield and environmental damage.
% State your goal: A decision support system for farmers.

\section{Problem Definition}
% Formalize the problem: 
% "Given historical weather W, crop health H, and inputs I, predict Yield Y and optimize I for max Y."
% Stakeholders: Farmers, Policymakers.

\section{Data Acquisition \& Preparation}
\label{sec:data}
% This is a critical section. Mention all 4 sources.

\subsection{Data Sources}
\begin{itemize}
    \item \textbf{Agricultural Statistics:} Punjab Crop Reporting Service (2002-2015). Target variable: Wheat Yield (Tons/Acre).
    \item \textbf{Remote Sensing:} MODIS NDVI (MOD13A2) via Google Earth Engine. 16-day composites aggregated to seasonal means.
    \item \textbf{Climate Data:} NASA POWER API. Daily rainfall and temperature aggregated to seasonal totals.
    \item \textbf{Spatial Boundaries:} GeoJSON from Kaggle for district-level aggregation.
\end{itemize}

\subsection{Preprocessing \& Integration}
% Mention how you cleaned district names (regex) and merged on 'District_Year_ID'.
% Mention handling missing values or outliers.

\section{Methodology}
% Describe your ML Pipeline.
\subsection{Feature Engineering}
% List features: Fertilizer_Usage, Mean_NDVI, Total_Rainfall, Avg_Temp.

\subsection{Models}
% Mention you trained Baseline (Mean), Random Forest, and Gradient Boosting.

\section{Experimental Setup}
% Train/Test Split (80/20).
% Evaluation Metric: RMSE (Root Mean Squared Error).

\section{Results \& Analysis}

\subsection{Model Comparison}
% Insert a table here comparing RMSE of Baseline vs GB Model.

\subsection{Feature Importance}
% Insert your feature_importance.png here
\begin{figure}[htbp]
\centerline{\includegraphics[width=3in]{feature_importance.png}}
\caption{Impact of features on Wheat Yield.}
\label{fig:feat_imp}
\end{figure}

\section{Proof of Concept}
% Describe the Streamlit App.
% Mention the "What-If" Simulation.
% Insert a screenshot of your dashboard.

\section{Conclusion}
% Summarize the impact.

\bibliographystyle{IEEEtran}
\bibliography{references}

\end{document}